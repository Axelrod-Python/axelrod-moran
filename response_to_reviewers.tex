\documentclass[a4]{article}

\usepackage{hyperref}

\title{Response to reviews}

\begin{document}

\maketitle

\section{Comments from editor}

Thank you for the review of our paper. We are delighted by the reviews: the
second author feeling there are no modifications necessary and the second
pointing out clarifications which have given us the opportunity to improve the
manuscript.

\begin{quote}
	1. Please ensure that your manuscript meets PLOS ONE's style requirements,
	including those for file naming. The PLOS ONE style templates can be found at
	\url{http://www.journals.plos.org/plosone/s/file?id=wjVg/PLOSOne_formatting_sample_main_body.pdf}
	and
	\url{http://www.journals.plos.org/plosone/s/file?id=ba62/PLOSOne_formatting_sample_title_authors_affiliations.pdf}
\end{quote}

\begin{quote}
    2. Please remove your figures from within your manuscript file, leaving only
    the individual TIFF/EPS image files, uploaded separately.  These will be
    automatically included in the reviewers’ PDF.
\end{quote}

All files have been renamed as requested and otherwise, the document follows the
given LaTeX style template.

\section{Comments from first referee}

\begin{quote}
    1) The authors discussed about as many as 164 strategies, which were
    trained/generated via different algorithms. Since understanding how the
    strategies work plays an important role in understanding why some of them
    succeed, I strongly suggest the authors to introduce the strategies in a more
    detailed and clear way, especially those based on finite state machines as I
    find no reference of them yet some of them perform well in the comparisons.
\end{quote}

This is a well made point by the referee: ensuring the reader can understand
these is vital to the paper.
We have clarified the description of the strategies:

\begin{itemize}
    \item Highlight that the appendix contains references for each strategy
        including pointing to a recently published paper that goes in to details
        about the structure of these strategies.
    \item A more formal definition of a finite state machine explaining how they
        fit in the framework of the IPD. We also have included numerous references
        which describes these.
\end{itemize}

\begin{quote}
  2) During the training/screening process, strategies generated via different
  algorithms were trained according to different criteria to determine whether a
  strategy was well trained or not. As examples, some were trained to maximize
  their payoff while others were trained with the objective function of mean
  fixation probabilities of Moran processes. So I am curious about how different
  criteria influence the features of the strategies? Will the conclusion of this
  work change if some strategies are trained in terms of other criteria?
\end{quote}

The referee asks a good question here, indeed the conclusion would be different
if strategies were trained in terms of other criteria. We have added a sentence
to this effect at the conclusion section of the paper highlighting that training
with evolutionary criteria seems to be fundamental to the appearance of a
handshake mechanism.

\begin{quote}
  3) The concept `handshake mechanism' is essential to appreciate the results as
  well as conclusion of this work but is not well defined, hence I suggest the
  authors to explain this concept clearly.
\end{quote}

We have added a better description of the term handshake mechanism after it is
first mentioned in the introduction. We have also included a reference to
another paper that makes mention of this as well as briefly discussing the
biological analogy of `kin recognition`.

\begin{quote}
  4) In line 241-242, it looks weird that there is a new paragraph starting in
  the middle of the sentence `The top 16 (10\%) strategies are shown in Table 4
  and figures...'
\end{quote}

Thank you to the referee for pointing this out: it was an error and has been
fixed.

\begin{quote}
  5) There are not a few phrases and sentences that may make the readers
  confusing. One example is `This is also the N for which...' in line 256.
\end{quote}

Thank you for pointing this out. That particular sentence has been clarified and
a thorough edit of the paper has been undertaken identifying and modifying
further confusing sentences.

\section{Comments from second referee}


\begin{quote}
  Reviewer \#2: This manuscript provides a detailed numerical analysis of
  agent-based simulations of 164 complex and adaptive strategies for the IPD,
  using an online GitHub library. The authors presented a lot of insights and
  empirical results. In my opinion, this paper is excellent, and worth publishing
  in the present form.
\end{quote}

Thank you for the confidence in our work and for the time taken to review.

\end{document}
